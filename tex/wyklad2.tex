\documentclass[../main.tex]{subfiles}
\graphicspath{
    {"../img/"}
    {"img/"}
}

\begin{document}
\subsection{Formy dwuliniowe (a) i formy kwadratowe (b)}
    \begin{align*}
        &(a)\quad \phi(w,\tilde w) = \int_0^1 w(t)\tilde w'(t) dt \in \mathbb{R},\quad w,\tilde w\in \mathbb{R}_3[.], \phi: \mathbb{R}_5[.] \times \mathbb{R}_5[.] \to \mathbb{R}\\
        &(b)\quad \varphi(w) = \phi(w,w) = \int_0^1 w(t)w'(t)dt,\quad \varphi: \mathbb{R}_3[.]\to \mathbb{R}
    .\end{align*}

    \begin{definicja}
        Niech $\phi: V \times V \to \mathbb{F}$ będzie formą dwuliniową. Odwzorowanie $\varphi: V\to \mathbb{F}$ takie, że $\varphi(v) = \phi(v,v)$ nazywamy formą kwadratową związaną z $\phi$
    \end{definicja}
    \begin{przyklad}
        Formy kwadratowe na $V=\mathbb{R}^2$. Niech $A=\begin{bmatrix} a&b\\c&d \end{bmatrix} , \phi_A(x,\tilde x) = x^T A \tilde x = \left[ x_1,x_2 \right] \begin{bmatrix} a&b\\c&d \end{bmatrix} \begin{bmatrix} x_1\\x_2 \end{bmatrix} = ax_1 \tilde x_1 + bx_1 \tilde x_2 + cx_2 \tilde x_1 + dx_2 \tilde x_2$\\
        $\varphi_A(x) = \phi_A(x,x) = \left[ x_1,x_2 \right] \begin{bmatrix} a&b\\c&d \end{bmatrix} \begin{bmatrix} x_1\\x_2 \end{bmatrix} = ax_1^2 + (b+c)x_1x_2 + dx_2^2$
    \end{przyklad}
    Przypomnienie: \[
        \phi = \phi_a + \phi_s, \phi_a(v,\tilde v) = \frac{1}{2}(\phi(v,\tilde v) - \phi(\tilde v,v)), \phi_s(v,\tilde v) = \frac{1}{2}(\phi(v,\tilde v)+\phi(\tilde v,v))
    .\]
    Zauważmy $\varphi(v) = \phi_a(v,v) + \phi_s(v,v) = \phi_s(v,v)$
    \begin{stw}
        Jeżeli $\varphi,\phi,\phi_a,\phi_s$ - jak wyżej, to \[
            \phi_s(v,\tilde v) = \frac{1}{2}(\varphi(v+\tilde v) - \varphi(v) - \varphi(\tilde v)) \text{ - formuła polaryzacyjna! }
        .\]
    \end{stw}
    \begin{dowod}
        Obliczmy $\varphi(v+\tilde v) = \phi(v+\tilde v,v+\tilde v) = \phi(v,v) + \phi(\tilde v,\tilde v) + \phi(v,\tilde v) + \phi(\tilde v,v) = \varphi(v) + \varphi(\tilde v) + 2\phi_s(v,\tilde v) \quad\Box$
    \end{dowod}
    Uwaga: Powyższe stwierdzenie zadaje $1-1$ odpowiedniość między symetrycznymi formami dwuliniowymi a formami kwadratowymi.
    \begin{przyklad}
        $\varphi:\mathbb{R}^2\to\mathbb{R}$ - forma kwadratowa.\\
        \[
        \varphi\begin{bmatrix} x_1\\x_2 \end{bmatrix} = ax_1^2 + bx_1x_2 + cx_2^2
        .\]
        $\underset{\lambda\in\mathbb{R}}{\forall} $ rozważmy $\phi_\lambda(x,\tilde x) = x^T \begin{bmatrix} a&\frac{b-\lambda}{2}\\\frac{b+\lambda}{2}&c \end{bmatrix} \tilde x$. Zauważmy, że $\varphi(x) = \phi_\lambda(x,x)$, $\phi_0$ jest symetryczną formą dwuliniową oraz $\varphi(x) = \phi_0(x,x)$
    \end{przyklad}
    \begin{przyklad}
        $\varphi$ - forma kwadratowa i niech $\phi$ będzie  symetryczną formą dwuliniową zadaną przez $\varphi$. Macierzą formy $\varphi$ w bazie $\mathcal{E}$ definiujemy jako macierz $\phi $ w $\mathcal{E}$.
       \[
         rk \varphi \overset{\text{def}}{=} rk \phi
       .\]
       $\varphi$ niezdegenerowana gdy $\phi$ jest niezdegenerowana. Wracając do przykładu: $\mathcal{E}$ - baza standardowa $\mathbb{R}^2$, \[
           \left[ \varphi \right]_\mathcal{E} = \begin{bmatrix} a&\frac{b}{2}\\ \frac{b}{2}&c \end{bmatrix}
       .\]
    \end{przyklad}
    \begin{definicja}
        Mówimy, że baza $\mathcal{E}$ diagonalizuje formę kwadratową $\varphi$ jeżeli macierz $\left[ \varphi \right]_\mathcal{E} $ jest diagonalna.
    \end{definicja}
    \begin{przyklad}
        $\varphi(x) = x_1^2 + 4x_1x_2 + 3x_2^2$. Znaleźć bazę diagonalizującą.
        \[
            \varphi(x) = (x_1+2x_2)^2 - x_2^2 = 3(x_2+\frac{2}{3}x_1)^2 - \frac{1}{3}x_1^1
        .\]
        Rozważmy dwie formy liniowe na $\mathbb{R}^2$:\\
        \begin{align*}
            \psi_1(x) = x_1+2x_2\\
            \psi_2(x) = x_2
        .\end{align*}
        $\left[ \varphi \right] _\mathcal{E} = \begin{bmatrix} 1&0\\0&1 \end{bmatrix} $
        Wówczas $\varphi(x) = (\psi_1(x))^2 - (\psi_2(x))^2 = (\psi_1^2 - \psi_2^2)(x)$
        \[
            \mathcal{E}^* = \left( \psi_1 = \left[ 1,2 \right] , \psi_2 = \left[ 0,1 \right]  \right) , \mathcal{E} = \left( f_1 = \begin{bmatrix} 1\\0 \end{bmatrix} , f_2 = \begin{bmatrix} -2\\1 \end{bmatrix}  \right)
        .\]
    \end{przyklad}

    Notacja: Niech $\varphi_1, \varphi_2\in V^*$. Wówczas funkcja $\varphi: v\in V \to \varphi_1(v)\varphi_2(v)\in\mathbb{F}$ jest formą kwadratową.
    \[
        \phi(v,\tilde v) = \varphi_1(v)\varphi_2(\tilde v), \frac{1}{2} (\varphi_1(v)\varphi_2(\tilde v) + \varphi_2(v)\varphi_1(\tilde v)) = \phi_s(v,\tilde v)
    .\]

    Notacja: $\varphi\overset{\text{ozn}}{=}\varphi_1\varphi_2,\quad \phi = \varphi_1\bigotimes\varphi_2$. W szczególności $\phi_s = \frac{1}{2}(\varphi_1\bigotimes\varphi_2 + \varphi_2\bigotimes\varphi_1$

    Jeśli teraz $\varphi:V\to\mathbb{F}$ - dowolna forma kwadratowa oraz $\mathcal{E} = \left( e_1,\ldots,e_n \right) $ - baza $V$, $\mathcal{E}^* = \left( \psi_1,\ldots,\psi_n \right) $ - baza dualna,\\
    Macierz $\varphi$ w $\mathcal{E}: \left[ \varphi \right] _\mathcal{E} = \left[ a_{ij} \right] ,\quad a_{ij} = a_{ji}$,\\
    zachodzi $\varphi = \sum_{i,j} a_{ij} = \psi_i \psi_j$

    \begin{tw}
        (Lagrange'a)\\
        Dla każdej formy kwadratowej itnieje (co najmniej jedna) baza diagonalizująca
    \end{tw}
    \begin{dowod}
        $\mathcal{E}^* = \left( \psi_1,\ldots,\psi_n \right) , \varphi = \sum_{i,j} a_{ij} \psi_i \psi_j$, gdzie $a_{ij} = a_{ji}$.\\
        Przypuśćmy, że $a_{ij}\neq 0$ dla pewnego $i$, np. $i = 1$.\\
        Rozważmy formę liniową
        \[
        \tilde \psi_1 = \psi_1 + \frac{1}{a_{11}}\sum_{j\neq 1} a_{1j}\psi_j
        .\]
        Wówczas istnieje wsp. bij. $i,j=2,\ldots,n$ taka, że \[
        \sum_{i,j} a_{ij}\psi_i \psi_j = a_{11} \tilde \psi_1^2 + \sum_{i,j=2}^n b_{ij}\psi_i \psi_j
        .\]
        np.\[
            a_{11}\psi_1^2 + a_{12}\psi_1\psi_2 + a_{21}\psi_2\psi_1 + a_{22}\psi_2^2 = a_{11}(\psi_1 + \frac{q_{12}}{a_{11}}\psi_z)^2 + (a_{22} - \frac{a_{12}^2}{a_{11}})\psi_2^2
        .\]
    \end{dowod}
    \begin{przyklad}
        $V=\mathbb{R}^3$.
        \[
            \varphi(x) = x_1x_2 + x_2x_3 = \overset{y_1}{\left(\frac{x_1+x_2}{2}\right)^2} - \overset{y_2}{\left( \frac{x_1-x_2}{2} \right) ^2} + x_2\overset{y_3}{x_3}
        .\]
        \[
            x_2 = y_1-y_2, \varphi(x) = y_1^2 - y_2^2 + y_1y_3 - y_2y_3 = (y_1+\frac{y_3}{2})^2 - \frac{y_3^2}{4} - y_2^2 - y_2y_3 = (y_1+\frac{y_3}{2})^2 - (y_2+\frac{y_3}{2})^2
        .\]
        \[
            \psi_1 = y_1+\frac{y_3}{2} = \frac{x_1+x_2+x_3}{2}, \psi_2 = \frac{x_1-x_2+x_3}{2}, \psi_3 = \ldots
        .\]
        \begin{align*}
            &\psi_1 = \frac{1}{2}\left[ 1,1,1 \right] \\
            &\psi_2 = \frac{1}{2}\left[ 1,-1,1 \right] \\
            &\psi_3 \overset{\text{np.}}{=} \left[ 1,0,-1 \right]
        .\end{align*}
        \[
            \mathcal{E}^* = \left( \psi_1, \psi_2, \psi_3 \right) , \mathcal{E} = \left( f_1,f_2,f_3 \right) , \varphi = \psi_1^2 - \psi_2^2, \left( f_1,f_2,f_3 \right) = \begin{bmatrix} \frac{1}{2}&\frac{1}{2}&\frac{1}{2}\\ \frac{1}{2}&-\frac{1}{2}&\frac{1}{2}\\ 1&0&-1 \end{bmatrix} ^{-1}
        .\]
    \end{przyklad}




\end{document}
