\documentclass[../main.tex]{subfiles}
\graphicspath{
    {"../img/"}
    {"img/"}
}

\begin{document}
    \begin{definicja}
        Mówimy, że macierz $A\in M_n(\mathbb{F})$est diagonalizowalna,eżeli istnieje baza $\mathcal{E}$ przestrzeni $\mathbb{F}^n$ taka, że $[A]_\mathcal{E} = diag(\lambda_1,\ldots,\lambda_n)$,\\
        gdzie $\lambda_1,\ldots,\lambda_n\in\mathbb{F}$, to znaczy, że $A e_k = \lambda_k e_k$.\\
        Jeżeli $G = [id]_\mathcal{E}^{st}$, to
        \[
            \begin{bmatrix} \lambda_1&\ldots&0\\ \vdots&\ddots&\vdots\\ 0&\dots&\lambda_n  \end{bmatrix} = G^{-1}A G
        .\]
        Wniosek: Macierz $A$ est diagonalizowalna jeżeli $\mathbb{F}^n$ ma bazę złożoną z wektorów własnych $A$.
    \end{definicja}
    \begin{przyklad}
        (antyprzykład)\\
        $A = \begin{bmatrix} 1&1\\0&1 \end{bmatrix} $ nie jest macierzą diagonalizowalną.
        \[
            w_a = \det \begin{bmatrix} 1-\lambda&1\\0\\1-\lambda \end{bmatrix} = (1-\lambda)^2,\quad Sp(A) = \{1\}
        .\]
        \[
            ker(A - 1) = ker\begin{bmatrix} 0&1\\0&0 \end{bmatrix} = \left< \begin{bmatrix} 1\\0 \end{bmatrix} \right>
        .\]
        Czyli $\mathbb{C}^2$ nie posiada bazy złożonej z wektorów własnych $A$.\\
        $\lambda = 1, n_1 = 2. V_\lambda = ker(A-\lambda 1)^{n_1} = ker(A-1)^2 = ker \begin{bmatrix} 0&1\\0&0 \end{bmatrix}^2$
    \end{przyklad}
    \begin{tw}
        $V$ - przestrzeń wektorowa nad $\mathbb{C}$. Ustala się endomorfizm $A\in L(V)$. $Sp(A) = \left\{ \lambda_1,\ldots,\lambda_k \right\} $ i niech
        \[
            w_A(\lambda) = \prod_{i=1}^{k} (\lambda_i - \lambda)^{n_i}
        .\]
        Zdefiniujmy $V_i = ker(A-\lambda_i 1)^{n_i}$. Wówczas $A V_i \subset V_i, V=\bigoplus V_i, dim V_i = n_i$
    \end{tw}
    Wniosek: Niech $\mathcal{E}$ będzie bazą $V$ zgodną z rozkładem $V = \bigoplus V_i$, to znaczy pierwsze $n_i$ wektorów $V$ jest bazą $V_i$, kolejne $n_2$ jest bazą $V_2$, itd. Wówczas istnieją macierze $A_i\in M_{n_i}(\mathbb{F})$ takie, że
    \[
        [A]_\mathcal{E} = \begin{bmatrix} A_1&&&0\\ &A_2&\\ &&\ddots&\\ 0&&&A_n \end{bmatrix}
    .\]
    \[
    A e_1 = \mu_1e_1 + \ldots \mu_{n_1}e_{n_1}
    .\]
    \begin{dowod}
        (równość $n_i = dim V_i$)\\
        Niech $w_i(\lambda) = \det (A_i - \lambda 1).$\\
        Wtedy
        \[
         w_A(\lambda) = \det ([A]_\mathcal{E} - \lambda 1) =
        \prod_{i=1}^{k} w_i(\lambda)
        .\]
        Niech $\lambda\in Sp(A_i).$ Zauważmy, że wówczas $\lambda\in Sp(A)$, to znaczy, że
        \[
            \underset{\lambda_j\in Sp(A)}{\exists}. \lambda = \lambda_j
        .\]
        Wtedy $V_i \cap V_j \neq \phi$, zatem $i =j$. Czyli $w_i(\lambda) = (\lambda_i - \lambda)^{dim V_i}.$ Zatem
        \[
            \prod_{i=1}^{k} (\lambda_i - \lambda)^{dim V_i} \implies n_i = dim V_i\quad\Box
        .\]
    \end{dowod}
    \begin{przyklad}
        Rozważmy równanie różniczkowe liniowe:
        \[
            \frac{d}{dt} \underset{v(t)}{\begin{bmatrix} x(t)\\y(t)\\z(t) \end{bmatrix}} =
            \underset{A}{\begin{bmatrix} 0&2&3\\1&3&5\\-1&-2&-4 \end{bmatrix}} \underset{\nabla(t)}{\begin{bmatrix} x(t)\\y(t)\\z(t) \end{bmatrix}}
        .\]
        \[
            v(t) = e^{A t}v(0) = v(0) \sum_{n=0}^{\infty} \frac{(t A)^n}{n!}
        .\]
    \end{przyklad}
    \begin{stw}
        Niech $f$ - funkcja analityczna oraz $w\in \mathbb{C}_n [.]$. Wtedy istnieje funkcja analityczna $q$ oraz wielomian $r\in\mathbb{C}_{n-1}[.]$ taki, że $f = wq + r$
    \end{stw}
    \begin{dowod}
        Indukcja ze względu na liczbę pierwiastków $w$.
        \[
            k=1:\quad w(\lambda) = (\lambda-\lambda_0)^n\\
            f(\lambda) = \sum_{k=0}^{\infty} \frac{f^{(k)}(\lambda_0)}{k!}(\lambda-\lambda_0)^k = (\lambda - \lambda_1)^n \underbrace{\sum_{k=n}^{\infty} \frac{f^{(n)}(\lambda_0)}{n!}(\lambda - \lambda_0)^{k-n}}_q + \underbrace{\sum_{k=0}^{n-1} \frac{f^{(k)}(\lambda_0)}{k!}(\lambda-\lambda_0)^k}_r
        .\]
        Indukcja:
        \[
            w(\lambda) = (\lambda - \lambda_1)^{n_1} \ldots (\lambda - \lambda_k)^{n_k} (\lambda-\lambda_{k+1})^{n_{k+1}}
        .\]
        Z indukcji istnieje $\tilde q$ - analityczna
        \[
            f(\lambda) = (\lambda-\lambda_1)^{n_1}\ldots(\lambda-\lambda_k)^{n_k}\tilde q(\lambda) + \tilde r(\lambda)
        .\] Gdzie $\tilde r\in \mathbb{C}_{n_1+\ldots+n_{k-1}}[.]$ oraz istnieje $\tilde \tilde q, \tilde \tilde r$ takie, że $\tilde q = (\lambda-\lambda_{k+1})^{n_{k+1}} \tilde \tilde q + \tilde \tilde r, \tilde \tilde r\in \mathbb{C}_{n_{k+1}-1}$
        Po wstawieniu $\tilde q: f(\lambda)=(\lambda-\lambda_1)^{n_1}\ldots(\lambda-\lambda_{k+1})^{n_{k+1}}\tilde \tilde q+r$, gdzie $r(\lambda) = \tilde r(\lambda) + \tilde \tilde r(\lambda) (\lambda-\lambda_1)^{n_1}\ldots(\lambda-\lambda_k)^{n_k}\quad\Box$
    \end{dowod}
    Zastosowanie powyższego stwierdzenia i twierdzenia Cayleya - Hamiltona.

    \begin{przyklad}
        Obliczyć $e^{A t}$ : rozważmy funkcję
        \[
            f: \lambda \to e^{\lambda t} = \sum_{n=0}^{\infty} \frac{(\lambda t)^n}{n!}
        .\]
        Chcemy obliczyć $f(A)$, gdzie $f$ jest funkcją analityczną (zadaną szeregiem).
        \[
            f(\lambda) = q(\lambda)w(\lambda) + r(\lambda); \quad w_A(A) = 0
        .\]
        Zatem
        \[
            f(A) = q(A)w(A)+ r(A) = r(A)
        .\]
        \[
            w_A(\lambda) = (1-\lambda)(1+\lambda)^2, Sp(A) = \left\{ 1,-1 \right\} ,n_1=1,n_2=2
        .\]
\[
    e^{t \lambda} = q(\lambda) (1-\lambda)(1+\lambda)^2 + a\lambda^2+b\lambda+c
.\]
Jak obliczyć $a,b,c\in\mathbb{R}$?
\begin{align*}
    &\lambda=1 &e^t = a+b+c\\
    &\lambda=-1 &e^{-t} = a-b+c\\
    &\frac{d}{d\lambda}e^{\lambda t} = &q'(\lambda)w(\lambda) + qw'(\lambda)+2a\lambda + b\implies\\
    &\lambda = -1 &te^{- t} = -2a+b
.\end{align*}
    \end{przyklad}
\end{document}
