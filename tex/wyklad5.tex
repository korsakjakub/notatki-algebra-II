\documentclass[../main.tex]{subfiles}
\graphicspath{
    {"../img/"}
    {"img/"}
}
\begin{document}
\[
    A\in \text{End} ( V ): V\to V
.\] wektory własne $v\in V-\{0\}$  $A v = \lambda v$
Wielomian charakterystyczny endomorfizmu \[
    w_A(\lambda) = \det (A - \lambda 1),\quad \lambda\in S_p(A)
.\]
\[
    V_\lambda = \ker (A - \lambda 1)
.\]
Rozwiązania równań różniczkowych wynika w pewnym sensie z następującego twierdzenia:
\begin{obserwacja}
    $u(t)$ - wielomian stopnia $n$, $u(t) = a_0+a_1 t + \ldots + a_n t^n$\\
    Endomorfizm postaci $a_0 1 + a_1 A + a_2 A^2 + \ldots + a_n A^n \in \text{End} (V)$ oznaczać będziemy $u(A)$.
    Własności
    \begin{align}
        &(u_1+u_2)(A) = u_1(A) + u_2(A)\\
        &(u_1 u_2)(A) = u_1(A) u_2(A)
    .\end{align}
\end{obserwacja}
\begin{przyklad}
    \[
    A = \begin{bmatrix}
    1 & 1\\
    1 & 0\end{bmatrix},
    w_A(\lambda) = \lambda^2 - \lambda - 1
    .\]
    \[
        w_A(A) =
        \begin{bmatrix}
        1 & 1\\
        1 & 0\end{bmatrix} ^2 -
        \begin{bmatrix}
        1 & 1\\
        1 & 0\end{bmatrix} -
        \begin{bmatrix}
        1 & 0\\
        0 & 1\end{bmatrix} =
        \begin{bmatrix}
        2 & 1\\
        1 & 1\end{bmatrix} -
        \begin{bmatrix}
        2 & 1\\
        1 & 1\end{bmatrix} = 0!!!
    .\]
\end{przyklad}

\begin{tw}
    (Cayleya - Hamiltona)
    \[
        \underset{A\in \text{End}(V)}{\forall} w_A(A) = 0
    .\]
\end{tw}
\begin{dowod}
    Niech $\mathcal{E}$ - baza $\mathcal{V}: \mathcal{A} = [ a_{ij} ] = \left[ A \right] _\mathcal{E}^\mathcal{E}$
    \[
        \left[ w(A) \right]_\mathcal{E}^\mathcal{E} = w\left( \left[ A \right] _\mathcal{E}^\mathcal{E} \right) \underset{w\in \mathbb{F}_k [x]}{\forall}
    .\]
    Zatem wystarczy udowodnić to dla macierzy $\mathcal{A}$

    Przypomnienie: macierz dopełnień algebraicznych \[
        \mathcal{A}^{D}\mathcal{A} = \det(\mathcal{A}) 1
    .\]
    W szczególności \[
        (\mathcal{A} - \lambda 1)^D (\mathcal{A} - \lambda 1 ) = \det(\mathcal{A} - \lambda 1) 1 = w_A(\lambda) 1
    .\]

    Uwaga: $n = \dim V$, to istnieją  $b_0,\ldots,b_{n-1} \in M_{n,n}(\mathbb{F})$ takie, że
    \begin{equation}\label{eq:100}
        (\mathcal{A} - \lambda 1)^D = b_0 + \lambda b_1 + \ldots + \lambda^{n-1} b_{n-1}
    \end{equation}

    Na przykład $($notacje: $\det \left[ a_{ij} \right] = \left| a_{ij} \right| )$

\[
\begin{bmatrix}
a_{11} - \lambda & a_{12} & a_{13}\\
a_{21} & a_{22} - \lambda & a_{23}\\
a_{31} & a_{32} & a_{33} - \lambda
\end{bmatrix}^D =
\begin{bmatrix}
    \begin{vmatrix}
        a_{22} - \lambda & a_{23}\\
        a_{32} & a_{33} - \lambda
    \end{vmatrix}
    & -
    \begin{vmatrix}
        a_{22} & a_{23}\\
        a_{32} & a_{33} - \lambda
    \end{vmatrix}
    & +
    \begin{vmatrix}
        &&\\
        &&\\
    \end{vmatrix}\\
    -
    \begin{vmatrix}
     &&\\
     &&\\
    \end{vmatrix}
    &
    \begin{vmatrix}
        a_{11} & a_{13}\\
        a_{31} & a_{33}
    \end{vmatrix}
    & -
    \begin{vmatrix}
      &&\\
      &&\\
    \end{vmatrix}\\
    \begin{vmatrix}
        &&\\
        &&\\
    \end{vmatrix}
    & -
    \begin{vmatrix}
       &&\\
       &&\\
    \end{vmatrix}
    &
    \begin{vmatrix}
        a_{11} - \lambda & a_{22}\\
        a_{21} & a_{22} - \lambda\\
    \end{vmatrix}

\end{bmatrix}
.\]

Oznaczenie $w_A(\lambda) = c_0 + c_1 \lambda + \ldots + c_n \lambda^n$
\ref{eq:100} oraz (123) \[
    (b_0+b_1 \lambda + \ldots + b_{n-1}\lambda^{n-1})(\mathcal{A} - \lambda 1) = c_0 1 + \lambda c_1 1 + \ldots + \lambda^n c_n 1
.\]

\begin{align*}
    \lambda^0 b_0 \mathcal{A} &= c_0 1 &|\mathcal{A}^0\\
    \lambda^1 b_1 \mathcal{A} - b_0 &= c_1 1 &|\mathcal{A}^1\\
    \lambda^{n-1} b_{n-1} \mathcal{A} - b_{n-2} &= c_{n-1} 1 &|\mathcal{A}^{n-1}\\
    \lambda^n - b_{n-1} &= c_n 1 &|\mathcal{A}^n\\
    + b_0 \mathcal{A} + (b_1 \mathcal{A}^2 - b_0 \mathcal{A}) + \ldots + b_{n-1}\mathcal{A}^n - b_{n-2}\mathcal{A}^{n-1} &= c_0 1 + c_1 \mathcal{A} + \ldots + c_n \mathcal{A}^n\\
    0 &= c_0 1 + c_1 \mathcal{A} + \ldots + c_n \mathcal{A}^n\Box
\end{align*}
\end{dowod}

\begin{przyklad}
    $x_n$ - ciąg Fibonacciego. $x_0 = 0, x_1 = 1$
    \[
    \begin{bmatrix}
    x_{n+1}\\
    x_{n}
\end{bmatrix} =
\begin{bmatrix}
1 & 1\\
1 & 0\end{bmatrix}^n
\begin{bmatrix}
1\\
0\end{bmatrix} =
\underbrace{
\begin{bmatrix}
1 & 1\\
1 & 0\end{bmatrix}^n}_{A}
\begin{bmatrix}
1\\
0\end{bmatrix}
    .\]
    \[
        w_A(\lambda) = \lambda^2 - \lambda - 1, u(\lambda) = \lambda^n
    .\]
    \[
        \lambda^n = u(\lambda) = q(\lambda)(\lambda^2 - \lambda - 1) + \underset{a\lambda+b_1}{r(\lambda)} \implies A^n = a A + b 1
    .\]
    Wyznaczamy $a$ i $b$:\\
    wartości własne wielomianu charakterystycznego: $\lambda_{+} = \frac{1+\sqrt{5}}{2}, \lambda_{-} = \frac{1-\sqrt{5}}{2}$ \\
    \begin{align*}
        \lambda_{+}^n = a \lambda_{+} + b_1\\
        \lambda_{-}^n = a \lambda_{-} + b_1
    .\end{align*} \[
        \implies a = \frac{1}{\sqrt{5} } \left ( \left (\frac{1+\sqrt{5} }{2})^2 - (\frac{1-\sqrt{5} }{2} \right )^2\right ), b = \ldots
    .\]
    \[
    \begin{bmatrix}
    x_{n+1}\\
    x_{n}
    \end{bmatrix} =
    a \begin{bmatrix}
    1\\
    1\end{bmatrix} + b \begin{bmatrix}
    1\\
    0\end{bmatrix} =
    \begin{bmatrix}
    a+b\\
a\end{bmatrix} \implies x_n = \frac{1}{\sqrt{5} }\left( \left( \frac{1+\sqrt{5}}{2}  \right)^n - \left( \frac{1-\sqrt{5} }{2} \right) ^n  \right)
    .\]
\end{przyklad}

Założenie: $\mathbb{F}\in \mathbb{C}, V$ nad  $\mathbb{C}$.\\
Ustalmy $A\in \text{End} (V), sp(A) = \left\{ \lambda_1, \ldots, \lambda_r \right\} $
\[
    w_A(\lambda) = \prod_{i=1}^r (\lambda_i - \lambda)^{n_i}, w_j(\lambda) = \prod_{i\neq j}^r (\lambda_i - \lambda)^{n_i}
.\]
Własności:\\
a) $j_1 \neq j_2$, to $\underset{u\in \mathbb{C}_m [.]}{\exists},\quad w_{j_1}(\lambda) w_{j_2}(\lambda) = u(\lambda)w_A(\lambda)$\\
b) $NWD(w_1,\ldots,w_r) = 1 \implies \underset{v_1,\ldots,v_r \in \mathbb{C}[.]}{\exists} 1 = v_1 w_1+ \ldots + v_r w_r$

Zdefiniujmy \[
    P_j = v_j(A) w_j(A),\quad j=1,\ldots,r
.\]
Własności rodziny $\left\{ P_1,\ldots,P_r \right\} $ \\
(i) $\sum_{j=1}^{r} {P_j = 1}, \\
(ii) j_1 \neq j_2:\quad P_{j_1} P_{j_2} = v_{j_1}(A) v_{j_2}(A) w_{j_1}(A) w_{j_2}(A)$ \\
(iii) $P_i^2 = P_i \sum_{j=1}^{r} P_j = P_i$ \\
(iv) niech $V_i = im P_i$. Wówczas $V = \overset{r}{\underset{i = 1}{\bigoplus}} V_i$\\

\[
    v = P_1 v + \ldots + P_r v \text{ i jeżeli } v\in V_{j_1} \bigcap V_{j_2} \implies P_{j_1} v = P_{j_1}P_{j_2} v = 0
.\]
(v) $V_j$ jest niezmiennicze na działanie $A$, gdyż $A P_j = A v_j(A) w_j (A) = v_j (A) w_j (A) A = P_i A$\\
a zatem jeżeli $v\in V_j$, to $A v = A P_j v = P_j A v \in V_j$\\
(vi)  $v_j = \ker \left( (A - \lambda_j 1)^{n_j} \right).$ $v\in v_j(a) w_j(A) v \implies (A - \lambda_1 1)^{n_j}v = v_j(A)w_j(A)(A - \lambda_j 1)^{n_j} = 0 \implies v\in \ker(A - \lambda_r 1)^{n_j}$\\
$(A-\lambda_j 1)^{n_j} v = 0 \implies v = P_1 v + \ldots + P_j v + \ldots + P_r v = P_j v \subset V, i\neq j, P_i v = s_i(A) (A - \lambda_j 1)^{n_j} v = 0$ dla każdego $s_j\in \mathbb{C}[.]$\\
(vii) $\dim V_j = n_j$

\begin{definicja}
    Przy powyższych oznaczeniach $v_j = \ker (A - \lambda_j 1)^{n_j}$ nazywamy podprzestrzenią pierwiastkową $A$
\end{definicja}

\begin{tw}
    O rozkładzie na podprzestrzenie pierwiastkowe
\end{tw}

\end{document}
