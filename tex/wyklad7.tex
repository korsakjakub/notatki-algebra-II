\documentclass[../main.tex]{subfiles}
\graphicspath{
    {"../img/"}
    {"img/"}
}

\begin{document}
    $V$ - wektorowa nad $\mathbb{R}$ lub $\mathbb{C}$. Na tej przestrzeni mamy iloczyn skalarny $\left< v_1 | v_2 \right> \in \mathbb{F}$. Wektory ortogonalne: $v_1 \perp v_2$, jeśli $\left< v_1 | v_2 \right> = 0$
    \begin{przyklad}
        na przestrzeni $\mathbb{C}^n$ wprowadzamy iloczyn skalarny $\left< u|w \right> = \sum_{i=1}^n \overline{u}_i w_i$, gdzie $\overline{u}$ - sprzężenie zespolone.
    \end{przyklad}
    Mówimy, że baza $\mathcal{E} = \left\{ v_1,\ldots,v_n \right\} $ przestrzeni $V$ jest ortonormalna, gdy $\left< v_i|v_j \right> = 0, i\neq j$.\\
    Notacja $\Vert v \Vert = \left< v|v \right>^{\frac{1}{2}}$ - długość wektora $v$.

    \begin{stw}
        Jeśli $\mathcal{E}$ jest bazą ortonormalną oraz $v = \sum_{i=1}^n \alpha_i v_i$, to $\alpha_i = \left< v_i | v \right>$.
    \end{stw}
    \begin{dowod}
        $\left< v_i | v \right> = \left< v_i | \sum_{j=1}^n \alpha_j v_j \right> = \sum_j \alpha_j \left< v_i | v_j \right> = \alpha_i$
    \end{dowod}
    \textit{Uwaga: Układ wektorów ortonormalnych jest liniowo niezależny.}\\
    $f_1, \ldots, f_k$ - układ ortonormalny: $\sum_{j=1}^k \alpha_j f_j = 0$, to $\alpha_i = \left< f_i | \sum_{j=1}^k \alpha_j f_j \right> = 0$
    \subsection{Ortogonalizacja Gramma-Schmidta}
    Niech $\left\{ e_1,\ldots,e_n \right\} $ będzie bazą $V$.\\
    Definiujemy \underline{indukcyjnie} układ (niezerowych) wektorów:\\
    $f_1 = e_1$; $f_1,\ldots,f_k$ - mamy to \[f_{k+1} = e_{k+1} - \sum_{j=1}^k \frac{\left< f_j | e_{k+1} \right>f_j}{\Vert f_j \Vert ^2}.\]
\textit{Uwaga: (1)}\\
$\left<f_1,\ldots,f_k \right> = \left<e_1,\ldots,e_k \right>$
    \begin{dowod}
        (indukcyjny)\\
        Dla $k=1$ - oczywiste.\\
        $k\implies k+1:$
        \[
            \left<f_1,\ldots,f_{k+1} \right> = \left<e_1,\ldots,e_k,f_{k+1} \right> = \left<e_1,\ldots,e_{k+1} \right>
        .\]
        W szczególności $\underset{k=1,\ldots,n}{\forall} f_k \neq 0$
    \end{dowod}

        \textit{Uwaga (2)}\\
        $f_i \perp f_j$ dla $i \neq j$.
        \begin{dowod}
            (indukcyjny)\\
            Przypuśćmy, że $i<j$.\\
            \begin{align*}
                &\left< f_i | f_j \right> = \left< f_i | e_j - \sum_{l=1}^{j-1} \frac{\left< f_l | e_j \right> f_l}{\Vert f_l \Vert ^2} \right> =\\
                &=\left<f_i|e_j \right> - \sum_{l=1}^{j-1} \frac{\left<f_l|e_j \right>}{\Vert f_l \Vert ^2} \left<f_i | f_l \right> =\\
                &=  \left<f_i|e_j \right> - \frac{\left<f_i|e_j \right>}{\Vert f_i \Vert ^2} \left<f_i | f_i \right> = 0
            .\end{align*}
            Kładąc $h_i = \frac{f_i}{\Vert f_i \Vert }$, dostaję bazę ortonormalną $\left\{ h_1,\ldots,h_n \right\} \quad\Box$
        \end{dowod}
        \begin{przyklad}
            Rozważamy przestrzeń wielomianów $V = \mathbb{R}[\times] = \left\{ \alpha_0 + \alpha_1x : \alpha_i \in \mathbb{R} \right\} $.\\
            $\mathcal{F} = \left\{ 1,x \right\} $, $\left<v_1|v_2 \right> = \int_0^1 v_1(x)v_2(x)dx.$ $\quad v_1 = \alpha_1 + \alpha_2x, v_2 = \beta_1 + \beta_2x$.\\
$\left<1|x \right> = \int_0^1 1 \cdot x dx = \frac{1}{2}$ \\
$f_1 = 1,\quad f_2 = x - \frac{\left<1|x \right> \cdot 1}{\Vert 1 \Vert ^2} = x-\frac{1}{2} \implies f_1\perp f_2$.\\
Czy $h_1$ jest unormowane? $h_1 = \frac{1}{\Vert 1 \Vert } = 1 = f_1$.\\
$h_2 = \frac{f_2}{\Vert f_2 \Vert } = \sqrt{12} (x-\frac{1}{2})$. $\quad \Vert f_2 \Vert ^2 = \int_0^1 (x-\frac{1}{2})^2 dx = \int_0^1 (x^2 - x + \frac{1}{4})dx = \frac{1}{12}$
        \end{przyklad}

        \subsection{Rzut ortogonalny}
        Ustalmy podprzestrzeń $E$ przestrzeni $V$. Niech $E^\perp = \left\{ v\in V: \underset{e\in E}{\forall} v\perp e \right\}, E^\perp $ - jest poprzestrzenią wektorową.
        Zauważmy, że $E \cap E^\perp = \{0\}: v\in E \cap E^\perp$, to $v\perp v: \left<v|v \right> = 0$.\\
        Ponadto, $E+E^\perp = V$. Ustalmy bazę ortonormalną podprzestrzeni $E = \left\{ e_1,\ldots,e_k \right\} .$\\
        Wtedy $v = \underbrace{\sum_{i=1}^k \left<e_i | v \right>e_i}_{\in E} + \left(v- \sum_{i=1}^k \left<e_i|v\right>e_i\right)$.\\
        Zauważmy $\left<e_l | v - \sum_{i=1}^k \left<e_i|v \right>e_i\right> = \left<e_l | v \right> - \left<e_l | v \right> = 0$.
        W takim razie
        \[
            v - \sum_{i=1}^k \left<e_i|v \right>e_i \in E^\perp
        .\]
        Wniosek: $V = E \bigoplus  E^\perp$. Rzut na $E$ wzdłuż $E^\perp$ nazywamy rzutem ortogonalnym na $E$ i oznaczamy $P_E$.\\
        Działa tak: $P_E v = \sum_{i=1}^k \left<e_i | v \right>e_i.$. $E^\perp$ nazywamy dopełnieniem ortogonalnym przestrzeni $E^{l=1}$
         \begin{stw}
             (Nierówność Cauchy-Schwartz)\\
             \[
                 \left| \left<v_1|v_2 \right> \right| \le \Vert v_1 \Vert \cdot \Vert v_2 \Vert
             .\]
         \end{stw}
         \begin{dowod}
             Niech $\alpha \in [0,2\pi]:\left<v_1|v_2 \right> = e^{i\alpha}\left| \left<v_1|v_2 \right> \right| $.\\
             Rozważmy funkcję $f: \mathbb{R}\to \mathbb{R}_{+}: f(t) = \left<t e^{i \alpha}v_1 - v_2 | te^{i\alpha}v_1-v_2 \right>$.
             \begin{align*}
                 &f(t) = \left<te^{i\alpha}v_1 | te^{i\alpha}v_1 \right> - \left<te^{i\alpha}v_1|v_2 \right> - \left<v_2|te^{i\alpha}v_1 \right> + \left<v_2|v_2 \right> = \\
                 &= t^2 \left<v_1|v_1 \right> - te^{-i\alpha}\left<v_1|v_2 \right> - te^{i\alpha} \left<v_2|v_2 \right> + \left<v_2|v_2 \right> \\
                 &= t^2 \left<v_1|v_1 \right> - 2t\left| \left<v_1|v_2 \right> \right| + \left<v_2|v_2 \right> \implies\\
                 &\implies \Delta = 4\left| \left<v_1|v_2 \right> \right| ^2 - 4 \left<v_1|v_1 \right>\left<v_2 |v_2\right>\le 0
             .\end{align*}
             $\quad\Box$
         \end{dowod}

         Wniosek (nierówność trójkąta)\\
         \[
         \underset{v_1,v_2\in V}{\forall} \Vert v_1+v_2 \Vert \le \Vert v_1 \Vert + \Vert v_2 \Vert
         .\]
         \begin{dowod}
             \begin{align*}
             &\Vert v_1+v_2 \Vert ^2 = \left< v_1+v_2 | v_1+v_2 \right> = \left<v_1|v_1 \right> + \left<v_2|v_1 \right> + \left<v_1|v_2 \right> =\\
             &= \Vert v_1 \Vert ^2 + 2 Re \left<v_1|v_2 \right> + \Vert v_2 \Vert ^2 \le \Vert v_1 \Vert ^2 + 2 \left| \left<v_1|v_2 \right> \right| +\Vert v_2 \Vert ^2 \le \\
             & \underset{\text{C.S.}}{\le} \Vert v_1 \Vert ^2 + 2 \Vert v_1 \Vert \Vert v_2 \Vert + \Vert v_2 \Vert ^2 = \left( \Vert v_1 \Vert +\Vert v_2 \Vert  \right) ^2
             .\end{align*}
             $\quad\Box$
         \end{dowod}

        Przestrzeń sprzężona do przestrzeni $V$ z iloczynem skalarnym.\\
        \begin{itemize}
            \item Niech $u\in V$. Wówczas $v\in V \to \left<u|v \right>\in \mathbb{F}$ jest elementem $V^*$, który oznaczamy  $\phi_u$.
                \[
                \left<\phi_u,v \right> = \left<u|v \right>
                .\]
        \end{itemize}
                \begin{przyklad}
                    $V = \mathbb{R}_3[\times], \phi_u(w) = \int_0^1 u(t)w(t)dt$
                \end{przyklad}
                Na odwrót:
                \begin{tw}
                    $\underset{\phi\in V^*}{\forall} \underset{u\in V}{\exists}! : \phi = \phi_u$
                \end{tw}
                \begin{dowod}
                    Jeżeli $\phi = 0$, to $u = 0$.\\
                    jeżeli $\phi \neq 0$, to $\ker \phi := E \not\subseteq V$. Wiemy, że $V = E \bigoplus E^\perp$.\\
                    Niech $u\in E^\perp - \{0\}: \left<\phi,u \right> = 1$.\\
                    Obliczmy $\left<u|v \right> = \left< u | v - \left< \phi,v \right>\cdot u + \left<\phi,v \right>\cdot u \right> = \left<u|\left<\phi,v \right>\cdot u \right> = \left<\phi,v \right>\Vert u \Vert ^2$
                \end{dowod}
                Podsumowując, $\left<\frac{u}{\Vert u \Vert ^2}|v \right> = \left<\phi,v \right> \implies \phi = \phi_{\frac{u}{\Vert u \Vert ^2}}$, co daje istnienie. Jedyność: jeśli $\phi_{u_1} = \phi_{u_2}$, to $\phi_{u_1-u_2} = 0$. Ale to oznacza, że $0 = \phi_{u_1-u_2}(u_1-u_2) = \left<u_1-u_2|u_1-u_2 \right> = \Vert u_1-u_2 \Vert ^2 \implies u_1=u_2 \quad\Box$



\end{document}
