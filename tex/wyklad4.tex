\documentclass[../main.tex]{subfiles}
\graphicspath{
    {"../img/"}
    {"img/"}
}

\begin{document}
Niech $\varphi: V\to \mathbb{F}$. mamy bazy $\mathcal{E}$ i $\mathcal{F}$. $\underbrace{([id]_{\mathcal{F}}^{\mathcal{E}})[\varphi]_\mathcal{E}[id]_{\mathcal{F}}^\mathcal{E} = [Q]_\mathcal{F}}_{\substack{\text{reguła transformacyjna dla}\\ \text{macierzy form kwadratowych}}}$

\subsection{Reguła transformacyjna macierzy odwzorowania liniowego}
\[
    A:V\to V\quad [id]_\mathcal{E}^\mathcal{F} [A]_\mathcal{E}^\mathcal{E} [id]_\mathcal{F}^\mathcal{E} = [A]_\mathcal{F}^\mathcal{F}, \quad [id]_\mathcal{E}^\mathcal{F} = \left( [id]_\mathcal{F}^\mathcal{E} \right) ^{-1}
.\]
Czy można zdiagonalizować macierz odwzorowania liniowego? Odpowiedź: następnych kilka wykładów.\\
Kończymy wątek o twierdzeniu Sylwestra:\\
niech $\varphi$ - dodatnio określona $D_i > 0$.
\begin{definicja}
    $\varphi$ jest ujemnie określona gdy $-\varphi$ jest dodatnio określona.
\end{definicja}
Wniosek: Forma $\varphi$ jest ujemnie określona gdy $(-1)^{2}D_i > 0$, gdzie $D_i = det\begin{bmatrix} \varphi_{11} & \ldots & \varphi_{1i}\\ \vdots & \ddots & \vdots \\ \varphi_{i1} & \ldots & \varphi_{ii} \end{bmatrix} \overset{\text{ozn}}{=} \begin{vmatrix} \varphi_{11} & \ldots & \varphi_{1i}\\ \vdots & \ddots & \vdots \\ \varphi_{i1} & \ldots & \varphi_{ii} \end{vmatrix}$
\begin{definicja}
    Odwzorowanie liniowe $A: V\to V$ nazywamy endomorfizmem przestrzeni $V $. $(L(V,V) \overset{\text{ozn}}{=} L(V))$
\end{definicja}
\subsection{Rzuty na podprzestrzenie}
\begin{przyklad}
    $\mathbb{R}^3 = V_1 \oplus W_1, \quad V_1 = \left< \begin{bmatrix} 1\\0\\0 \end{bmatrix} , \begin{bmatrix} 0\\1\\0 \end{bmatrix}  \right>, \quad W_1 = \left< \begin{bmatrix} 0\\0\\1 \end{bmatrix}  \right>$\\
    $\begin{bmatrix} x\\y\\z \end{bmatrix} = \begin{bmatrix} x\\y\\0 \end{bmatrix} + \begin{bmatrix} 0\\0\\z \end{bmatrix}, \quad P \begin{bmatrix} x\\y\\z \end{bmatrix} = \begin{bmatrix} x\\y\\0 \end{bmatrix}$. Zauważmy, że $P_1^2 = P_1$. $\left( [P_1]_{st}= \begin{bmatrix} 1&0&0\\0&1&0\\0&0&0 \end{bmatrix}  \right) $
\end{przyklad}
\begin{przyklad}
    Inny rozkład: $\mathbb{R}^3 = \left<\begin{bmatrix} 1\\0\\0 \end{bmatrix} , \begin{bmatrix} 0\\1\\0 \end{bmatrix}  \right> \bigoplus \left<\begin{bmatrix} 1\\1\\1 \end{bmatrix}  \right>$.
    \[
        \begin{bmatrix} x\\y\\z \end{bmatrix} = \begin{bmatrix} x-z\\y-z\\0 \end{bmatrix} + \begin{bmatrix} z\\z\\z \end{bmatrix}
    .\]
    \[
        P_2\left( \begin{bmatrix} x\\y\\z \end{bmatrix}  \right) = \begin{bmatrix} x-z\\y-z\\0 \end{bmatrix},\quad P_2^2 = P_2. \left([P_2]\right)_{st} = \begin{bmatrix} 1&0&1\\0&1&1\\0&0&0 \end{bmatrix}
    .\]
\end{przyklad}
Ogólniej: Jeżeli przestrzeń wektorowa  $U$ jest sumą prostą $V, W\subset U$, to operator rzutu na $V$ wzdłuż $W$ jest dany następującym wzorem: \[
Pu = v
.\] gdzie $u = v+w, v\in V, w\in W$. Łatwo sprawdzić, że  $P^2 = P, W = \ker P, im P = V$
 \begin{definicja}
     Endomorfizm $P\in L(U)$ nazywamy rzutem, gdy $P^2 = P$
\end{definicja}
\begin{stw}
    $P\in L(U), P^2 = P, W = im P, V = im P$. Wtedy $U = V \bigoplus W$ oraz $P$ jest rzutem na $V$ wzdłuż $W$.
\end{stw}
\begin{dowod}
    Weźmy $u\in U: u = Pu + (1-P)u \& Pu\in im P \& (1-P)u\in \ker P$, gdyż $P(1-P)u = (P-P^2)u = 0$. Czy  $im P \cap \ker P = \{0\}$?\\
    Jeśli  $u\in im P \& u\in \ker P$, to $\underset{x\in V}{\exists} u = Px = P Px = Pu = 0$
\end{dowod}
\begin{definicja}
    Jeżeli $A\in L(U)$ oraz $V\subset U$ jest podprzestrzenią taką, że $A V \subset V$, to mówimy, że jest $A$ - niezmiennicza.
\end{definicja}
Uwaga: Niech $V$ będzię niezmiennicze dla $A: U\to U, \mathcal{E}_0 = \left\{ e_1,...,e_k \right\} $ dla bazy V, $\mathcal{E}_1 = \left\{ e_1,...,e_k,e_{k+1},\ldots,e_n \right\} $ - baza $U \implies [A]_{\mathcal{E}_1} = \begin{bmatrix} a_{11} & \ldots & a_{1k} & \\ \vdots & & &* \\ a_{k_1} & \ldots & a_{kk} & \\ 0 & \ldots & 0 & \\ \vdots &  & \vdots & ** \\ 0 & \ldots & 0 &\end{bmatrix}$,\\
$*\in M_{k,n-k}(\mathbb{F}), **\in M_{n-k,n-k}(\mathbb{F}) $\\
Uwaga 2:
Przypuśćmy, że $U = V_1 \bigoplus V_2 \bigoplus \ldots \bigoplus V_l \quad \& A V_i \subset V_i, i\in 1,\ldots,l$. Wtedy istnieje baza $\mathcal{E}$ przestrzeni $U$ taka, że gdzie $B_i\in M_{n_i \times n_i}(\mathbb{F}) \quad \& \quad n_i = \dim V_i$
\[
    A = \begin{bmatrix} B_1 & 0 & \ldots & 0 \\ 0 & B_2 & & \vdots \\ \vdots & & \ddots & 0 \\ 0 & \ldots & 0 & B_l \end{bmatrix}
.\]
\[
    \mathcal{E} = \left\{ e_1,\ldots, e_{n_1}, e_{n_1+1}, \ldots, e_{n_1+n_2}, \ldots, \ldots, e_{n_1+n_2+\ldots+n_l} \right\}
.\]
\begin{definicja}
    Mówimy, że $0\neq u\in U$ jest wektorem własnym $A \in L(U)$, jeśli $A u = \lambda u$ dla pewnego skalara $\lambda_i \in \mathbb{F}$. Mówimy wówczas, że jest wartością własną $A$. Zbiór wartości własnych $A$ nazywamy spektrum $A$ i oznaczamy $sp(A) \subset \mathbb{F}$. Jeżeli $\lambda\in\mathbb{F}$, to $V_\lambda = \ker(A - \lambda 1)$ nazywamy podprzestrzenią własną dla $\lambda\in \mathbb{F}$.
\end{definicja}
Zauważmy $\lambda \in sp(A) \iff \ker(A - \lambda 1) \neq \{0\} \iff det(A-\lambda 1) = 0 \iff A - \lambda 1$ jest operatorem nieodwracalnym.\\
Uwaga: Jeśli $A\in L(V)$, to $\det\left( [A]_\mathcal{E} \right)  = \det \left( [A]_\mathcal{F} \right)  = \det(A)$, gdyż $\det[A]_{\mathcal{E}} = \det\left( \left( [id]_\mathcal{E}^\mathcal{F} \right) ^{-1} [A]_\mathcal{F}[id]_\mathcal{E}^\mathcal{F} \right) = \det\left( [A]_\mathcal{F} \right) $.\\
Operator $A$ jest odwracalny $\iff$ $[A]_\mathcal{E}$ - odwracalna $\iff$ $\det A \neq 0$
\begin{definicja}
    Wielomian $\lambda\in\mathbb{F}\to \det(A-\lambda 1) \in \mathbb{F}$ nazywamy wielomianem charakterystycznym operatora $A$, oznaczamy $w_A(\lambda)$
\end{definicja}
Wniosek: $sp A = \left\{ \lambda\in\mathbb{F}: w_A(\lambda) = 0 \right\} $.\\
\begin{przyklad}
    $A\in M_{2\times 2}(\mathbb{R}), A = \begin{bmatrix} 1&1\\1&0 \end{bmatrix} $. $sp A: w_A(\lambda) = \det \begin{bmatrix} 1-\lambda&1\\1&-\lambda \end{bmatrix} = \lambda^2 - \lambda - 1$\\
    Pierwiastki $w_A: \Delta = 1+4,\quad \lambda_1 = \frac{1-\sqrt{5} }{2}, \quad \lambda_2 = \frac{1+\sqrt{5} }{2}$\\
    Wektory własne: $V_{\lambda_1} = \ker\left(\begin{bmatrix} 1&1\\1&0 \end{bmatrix} - \lambda_1 \begin{bmatrix} 1&0\\0&1 \end{bmatrix}\right) = \ker \begin{bmatrix} \frac{1+\sqrt{5} }{2} & 1 \\ 1 & \frac{-1+\sqrt{5} }{2} \end{bmatrix} = \left<\begin{bmatrix} -1\\ \frac{1+\sqrt{5} }{2} \end{bmatrix}  \right> $.\\
    $V_{\lambda_2} = \ker \begin{bmatrix} \frac{1-\sqrt{5} }{2} & 1 \\ 1 & - \frac{1+\sqrt{5} }{2} \end{bmatrix} = \left<\begin{bmatrix} \frac{1+\sqrt{5} }{2}\\1 \end{bmatrix}  \right>$
\end{przyklad}
Ciąg Fibonacciego: $x_0=1=x_1, (1,1,2,3,5,8,13,21,\ldots)$
$x_{n+2} = x_{n+1}+x_n$. Znaleźć ogólny wyraz $x_n = ?$\\
    Wielomian charakterystyczny $\lambda^2 - \lambda - 1$. Zauważmy, że $\begin{bmatrix} x_{n+2}\\x_{n+1} \end{bmatrix} = \begin{bmatrix} 1&1\\1&0 \end{bmatrix} \begin{bmatrix} x_{n+1}\\x_n \end{bmatrix} = \ldots = A^{n+1}\begin{bmatrix} x_1\\x_0 \end{bmatrix} $
\end{document}
